\documentclass[UTF8]{ctexart}
\usepackage{geometry, CJKutf8}
\geometry{margin=1.5cm, vmargin={0pt,1cm}}
\setlength{\topmargin}{-1cm}
\setlength{\paperheight}{29.7cm}
\setlength{\textheight}{25.3cm}

% useful packages.
\usepackage{amsfonts}
\usepackage{amsmath}
\usepackage{amssymb}
\usepackage{amsthm}
\usepackage{enumerate}
\usepackage{graphicx}
\usepackage{multicol}
\usepackage{fancyhdr}
\usepackage{layout}
\usepackage{listings}
\usepackage{float, caption}

\lstset{
	basicstyle=\ttfamily, basewidth=0.5em
}

% some common command
\newcommand{\dif}{\mathrm{d}}
\newcommand{\avg}[1]{\left\langle #1 \right\rangle}
\newcommand{\difFrac}[2]{\frac{\dif #1}{\dif #2}}
\newcommand{\pdfFrac}[2]{\frac{\partial #1}{\partial #2}}
\newcommand{\OFL}{\mathrm{OFL}}
\newcommand{\UFL}{\mathrm{UFL}}
\newcommand{\fl}{\mathrm{fl}}
\newcommand{\op}{\odot}
\newcommand{\Eabs}{E_{\mathrm{abs}}}
\newcommand{\Erel}{E_{\mathrm{rel}}}

\begin{document}
	
	\pagestyle{fancy}
	\fancyhead{}
	\lhead{田佳豪, 3230105412}
	\chead{数据结构与算法第七次作业}
	\rhead{Nov.29th, 2024}
	
	\section{测试程序的设计思路}
	
	1.在HeapSort.h文件中,我使用了C++ STL中的工具,先使用makeheap()函数建堆,将序列生成为一个堆序列;之后用一个for循环和popheap()函数,将堆顶元素(即最大元素)放置于堆尾然后,每次for循环后都排除被放于堆尾的元素即可
	
	2.在test()函数中,我首先设计了一个随机数生成器,同于生成一定范围内的整型或者浮点型的随机数
	
	3.我设计了一个check()函数,通过检查排序后是否为升序序列来测试HeapSort的准确性
	
	4.之后我建立了四个序列,分别为长度1000000的随机浮点序列、升序浮点序列、降序浮点序列、有重复数字的随机整数序列来进行测试,之后进行克隆
	
	5.使用std::sortheap函数与我自己写的HeapSort函数进行时间测试,并且check函数检查排序正确性
	\section{测试的结果}
	
	1.check函数检查HeapSort排序结果无误
	
	2.测试时间表格如下(测试时间均小于3000ms):
	
	\begin{tabular}{|c|c|c|}
	\hline
	sequence & my heapsort time & std::sortheap time \\ \hline
	random sequence & 668 & 794 \\ \hline
	ordered sequence & 494 & 838 \\ \hline
	reverse sequence & 843 & 532 \\ \hline
	repetitive sequence & 742 & 713 \\ \hline
	
	\end{tabular}

	输出代码块如下:
	\begin{lstlisting}
	668ms
	1
	494ms
	1
	843ms
	1
	742ms
	1
	794ms
	838ms
	532ms
	713ms
	\end{lstlisting}
	
	
		
	我的一些分析:我的HeapSort和std::sortheap()函数在排序长度为1000000的四种序列中的表现相近,均在500ms~1000ms之间
	根据了解,std::sortheap也是通过调用这两个函数完成的,因此我的HeapSort的时间复杂度:建堆函数的时间复杂度为O(n)
	而popheap()的时间复杂度为O(logn),因为需要对n个元素进行pop操作,最终时间复杂度为O(nlogn) 
	所以总体的时间复杂度为O(nlogn)
	
	测试结果一切正常。
	
	我用 valgrind 进行测试,发现没有发生内存泄露。
	
	
\end{document}

%%% Local Variables: 
%%% mode: latex
%%% TeX-master: t
%%% End: 
