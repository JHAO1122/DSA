\documentclass[UTF8]{ctexart}
\usepackage{geometry, CJKutf8}
\geometry{margin=1.5cm, vmargin={0pt,1cm}}
\setlength{\topmargin}{-1cm}
\setlength{\paperheight}{29.7cm}
\setlength{\textheight}{25.3cm}

% useful packages.
\usepackage{amsfonts}
\usepackage{amsmath}
\usepackage{amssymb}
\usepackage{amsthm}
\usepackage{enumerate}
\usepackage{graphicx}
\usepackage{multicol}
\usepackage{fancyhdr}
\usepackage{layout}
\usepackage{listings}
\usepackage{float, caption}

\lstset{
	basicstyle=\ttfamily, basewidth=0.5em
}

% some common command
\newcommand{\dif}{\mathrm{d}}
\newcommand{\avg}[1]{\left\langle #1 \right\rangle}
\newcommand{\difFrac}[2]{\frac{\dif #1}{\dif #2}}
\newcommand{\pdfFrac}[2]{\frac{\partial #1}{\partial #2}}
\newcommand{\OFL}{\mathrm{OFL}}
\newcommand{\UFL}{\mathrm{UFL}}
\newcommand{\fl}{\mathrm{fl}}
\newcommand{\op}{\odot}
\newcommand{\Eabs}{E_{\mathrm{abs}}}
\newcommand{\Erel}{E_{\mathrm{rel}}}

\begin{document}
	
	\pagestyle{fancy}
	\fancyhead{}
	\lhead{田佳豪, 3230105412}
	\chead{数据结构与算法第四次作业}
	\rhead{Oct.16th, 2024}
	
	\section{测试程序的设计思路}
	
	1.我初始化了一个名为List的链表,之后用clear()清空此链表,测试初始化和清空操作。
	
	2.我声明了一个空链表叫lst,并判断是否为空,测试判断链表是否为空操作。
	
	3.我使用pushback配合for循环以及pushfront给链表赋值,测试左值右值插入功能,同时可以测试insert()。	
	
	4.我调用了popfront和popback来测试其功能,同时测试erase。
	
	5.我调用back和front,来验证返回数据是否准确以及可修改性。
	
	6.我调用size返回lst的大小来测试其功能。
	
	7.我用std::move()移动构造链表lst2,并输出lst2进行检验。
	
	8.我重新定义了一个lst3并初始化,之后用赋值lst4验证拷贝构造函数,输出lst3和lst4检验。
	
	9.我使用一个for循环和指针自增操作来检验前置自增运算符和后置自增运算符。
	\section{测试的结果}
	1.初始化链表list后clear,输出链表发现为空,测试成功。
	
	2.声明一个链表lst,用empty()检验,输出1,发现为空,测试成功。
	
	3.pushback for i from 0 to 9 后pushfront 100,输出100 0 ... 9 测试成功。 
	
	4.popfront和popback后输出0 ... 8,第一和最后一个元素被删除,测试成功。
	
	5.back后输出8,back测试成功;front成功将第一个修改,输出100 1 ... 8测试成功。
	
	6.size返回链表大小,输出值为8,符合,测试成功;
	
	7.调用移动构造函数后输出lst2,与lst1保持一致,测试成功。
	
	8.调用拷贝构造函数后输出lst3与lst4,发现二者一致,并将lst4最后一个数据删除后lst3不变,成功深拷贝。
	
	9.通过输出lst3和lst4验证前置自增运算符和后置自增运算符,输出结果无误,测试成功。
	
	输出代码块如下:
	\begin{lstlisting}
	
	
	1
	100	0	1	2	3	4	5	6	7	8	9	
	0	1	2	3	4	5	6	7	8	
	8
	100	1	2	3	4	5	6	7	8	
	9
	100	1	2	3	4	5	6	7	8	
	1	2	3	4	5	
	1	2	3	4	5	
	1	2	3	4	5	
	1	2	3	4	
	
	\end{lstlisting}
	
		
	测试结果一切正常。
	
	我用 valgrind 进行测试,发现没有发生内存泄露。
	
	
\end{document}

%%% Local Variables: 
%%% mode: latex
%%% TeX-master: t
%%% End: 

