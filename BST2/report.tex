\documentclass[UTF8]{ctexart}
\usepackage{geometry, CJKutf8}
\geometry{margin=1.5cm, vmargin={0pt,1cm}}
\setlength{\topmargin}{-1cm}
\setlength{\paperheight}{29.7cm}
\setlength{\textheight}{25.3cm}

% useful packages.
\usepackage{amsfonts}
\usepackage{amsmath}
\usepackage{amssymb}
\usepackage{amsthm}
\usepackage{enumerate}
\usepackage{graphicx}
\usepackage{multicol}
\usepackage{fancyhdr}
\usepackage{layout}
\usepackage{listings}
\usepackage{float, caption}

\lstset{
	basicstyle=\ttfamily, basewidth=0.5em
}

% some common command
\newcommand{\dif}{\mathrm{d}}
\newcommand{\avg}[1]{\left\langle #1 \right\rangle}
\newcommand{\difFrac}[2]{\frac{\dif #1}{\dif #2}}
\newcommand{\pdfFrac}[2]{\frac{\partial #1}{\partial #2}}
\newcommand{\OFL}{\mathrm{OFL}}
\newcommand{\UFL}{\mathrm{UFL}}
\newcommand{\fl}{\mathrm{fl}}
\newcommand{\op}{\odot}
\newcommand{\Eabs}{E_{\mathrm{abs}}}
\newcommand{\Erel}{E_{\mathrm{rel}}}

\begin{document}
	
	\pagestyle{fancy}
	\fancyhead{}
	\lhead{田佳豪, 3230105412}
	\chead{数据结构与算法第六次作业}
	\rhead{Oct.16th, 2024}
	
	\section{测试程序的设计思路}
	1.我首先写了一个height()函数,如果节点为空返回0,否则返回节点的高度。
	
	2.之后我开始写balance()函数,分四种情况讨论:
	
	如果是左孩子的左子树过高,则进行一次左边的单旋转;
	
	如果是右孩子的右子树过高,则进行一次右边的单旋转;
	
	如果是左孩子的右子树过高,则进行一次左-右的双旋转;
	
	如果是右孩子的左子树过高,则进行一次右-左的双旋转;
	
	3.左边的单旋转:将左孩子的右子树给父节点,随后将原左子节点变为原父节点的父节点,之后更新节点。
	
	4.右边的单旋转:将右孩子的左子树给父节点,随后将原右子节点变为原父节点的父节点,之后更新节点。
	
	5.左-右双旋转:对父节点的左子节点进行一次单旋转,再对父节点进行一次单旋转。
	
	6.右-左双旋转:对父节点的右子节点进行一次单旋转,再对父节点进行一次单旋转。

	7.更改remove()函数,使得树以AVL tree的方式平衡。
	
\end{document}

%%% Local Variables: 
%%% mode: latex
%%% TeX-master: t
%%% End: 
