\documentclass[UTF8]{ctexart}
\usepackage{geometry, CJKutf8}
\geometry{margin=1.5cm, vmargin={0pt,1cm}}
\setlength{\topmargin}{-1cm}
\setlength{\paperheight}{29.7cm}
\setlength{\textheight}{25.3cm}

% useful packages.
\usepackage{amsfonts}
\usepackage{amsmath}
\usepackage{amssymb}
\usepackage{amsthm}
\usepackage{enumerate}
\usepackage{graphicx}
\usepackage{multicol}
\usepackage{fancyhdr}
\usepackage{layout}
\usepackage{listings}
\usepackage{float, caption}

\lstset{
	basicstyle=\ttfamily, basewidth=0.5em
}

% some common command
\newcommand{\dif}{\mathrm{d}}
\newcommand{\avg}[1]{\left\langle #1 \right\rangle}
\newcommand{\difFrac}[2]{\frac{\dif #1}{\dif #2}}
\newcommand{\pdfFrac}[2]{\frac{\partial #1}{\partial #2}}
\newcommand{\OFL}{\mathrm{OFL}}
\newcommand{\UFL}{\mathrm{UFL}}
\newcommand{\fl}{\mathrm{fl}}
\newcommand{\op}{\odot}
\newcommand{\Eabs}{E_{\mathrm{abs}}}
\newcommand{\Erel}{E_{\mathrm{rel}}}

\begin{document}
	
	\pagestyle{fancy}
	\fancyhead{}
	\lhead{田佳豪, 3230105412}
	\chead{数据结构与算法第五次作业}
	\rhead{Oct.16th, 2024}
	
	\section{测试程序的设计思路}
	\begin{verbatim}
			void remove(const Comparable &x, BinaryNode *&t) {
			BinaryNode *c = t; //子节点
			BinaryNode *p = nullptr; //父节点
			while (c != nullptr) {//不为空树
				if (x < c->element) {
					p = c;
					c = c->left;
				} else if (x > c->element) {
					p = c;
					c = c->right;//查找待删除节点
				} else { 
					if (c->left != nullptr && c->right != nullptr) { 
						BinaryNode *minNode = detachMin(c->right);//找到右子树的最小节点
						c->element = minNode->element; 
						delete minNode; 
					} else { 
						BinaryNode *oldNode = c;
						c = (c->left != nullptr) ? c->left : c->right;//实现节点的替换和删除 
						if (p == nullptr) { 
							t = c; 
						} else if (p->left == oldNode) {
							p->left = c; //如果原查找到的节点在父节点的左边,则将现子节点连接到原父节点左边
						} else {
							p->right = c; //如果原查找到的节点在父节点的右边,则将现子节点连接到原父节点右边
						}
						delete oldNode; //释放内存
					}
					return; 
				}
			}
		}
		
		BinaryNode *detachMin(BinaryNode *&t) {
			if (t == nullptr) return nullptr;
			BinaryNode *temp1 = nullptr; //父节点
			BinaryNode *temp2 = t;//子节点 
			while (temp2->left != nullptr) {
				temp1 = temp2;
				temp2 = temp2->left;//循环找到最小值
			}
			if (temp1 != nullptr) {
				temp1->left = temp2->right; //实现节点连接更新,将待删除节点的右数连接到父节点左边
			} else {
				t = temp2->right; 
			}
			return temp2; 
		}
	\end{verbatim}
	 
	1.我首先创建了一个detachMin函数,用于查找待删除节点有两个子节点的情况时的右子树的最小值,
	更新节点,删除后返回;
	
	2.对于待删除节点只有一个或者没有子节点的情况,我在remove函数中直接处理;
	
	\section{测试的结果}
	
	测试结果一切正常。
	
	1.我调用insert()函数依次输入10 5 15 3 7 12 18,得到树的遍历为3 5 7 10 12 15 18
	
	2.调用remove对待删除节点没有左右两个子节点的情况测试,remove(7)后树为3 5 10 12 15 18
	
	3.调用remove对待删除节点有左右两个子树的情况测试,remove(10)后树为3 5 12 15 18
	
	我用 valgrind 进行测试,发现没有发生内存泄露。
	
    输出代码块如下:
    \begin{verbatim}
    Tree after removing 7:
    3
    5
    10
    12
    15
    18
    Tree after removing 10:
    3
    5
    12
    15
    18
	\end{verbatim}
\end{document}

%%% Local Variables: 
%%% mode: latex
%%% TeX-master: t
%%% End: 
