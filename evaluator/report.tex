\documentclass[UTF8]{ctexart}
\usepackage{geometry, CJKutf8}
\geometry{margin=1.5cm, vmargin={0pt,1cm}}
\setlength{\topmargin}{-1cm}
\setlength{\paperheight}{29.7cm}
\setlength{\textheight}{25.3cm}

% useful packages.
\usepackage{amsfonts}
\usepackage{amsmath}
\usepackage{amssymb}
\usepackage{amsthm}
\usepackage{enumerate}
\usepackage{graphicx}
\usepackage{multicol}
\usepackage{fancyhdr}
\usepackage{layout}
\usepackage{listings}
\usepackage{float, caption}

\lstset{
	basicstyle=\ttfamily, basewidth=0.5em
}

% some common command
\newcommand{\dif}{\mathrm{d}}
\newcommand{\avg}[1]{\left\langle #1 \right\rangle}
\newcommand{\difFrac}[2]{\frac{\dif #1}{\dif #2}}
\newcommand{\pdfFrac}[2]{\frac{\partial #1}{\partial #2}}
\newcommand{\OFL}{\mathrm{OFL}}
\newcommand{\UFL}{\mathrm{UFL}}
\newcommand{\fl}{\mathrm{fl}}
\newcommand{\op}{\odot}
\newcommand{\Eabs}{E_{\mathrm{abs}}}
\newcommand{\Erel}{E_{\mathrm{rel}}}

\begin{document}
	
	\pagestyle{fancy}
	\fancyhead{}
	\lhead{田佳豪, 3230105412}
	\chead{数据结构与算法项目作业报告}
	\rhead{Nov.22th, 2024}
	
	\section{测试程序的设计思路}
	    
	\subsection{混合运算的计算器的设计}
	
	\subsubsection{transfer类}
	
	\paragraph{前期准备:}我设计了几个函数为我之后的设计调用函数做准备		
	
	1.构造函数,得到需要转化表达式的字符串
	
	2.isoperator()函数,判断是否为加减乘除的运算符的函数
	
	3.operatorrank()函数,比较各个运算符的优先级
	
	4.count()函数,返回在字符串某位置及之前的某字符的数量和
	
	\paragraph{islegal()函数:}
	
	我设计了布尔函数一个判断表达式是否合法的函数,如果出现包括运算符连续使用、括号不匹配、表达式以运算符开头或结尾、除数为0的情况,则返回0,否则中缀表达式合法,返回1
	
	\paragraph{change()函数}
	
	我设计了一个change()函数,可以将中缀表达式转化为后缀表达式并返回后者	
	
	1.首先声明一个运算符栈,和一个字符串,字符串作为输出
	
	2.如果遇到了数字,则直接输出到字符串中
	
	3.如果遇到了左括号,则直接压入栈中
	
	4.如果遇到了右括号,则把栈顶的运算符弹出,直到遇到左括号
	
	5.如果遇到了运算符,则比较它与栈顶运算符的优先级,如果前者小则弹出栈顶运算符
	
	\subsubsection{solution类}

	\paragraph{solution()函数}
	
	我设计了一个solution()函数,用于计算后缀表达式的值
	
	1.首先声明一个运算数栈
	
	2.开始遍历后缀表达式,遇到空格什么都不做
	
	3.如果遇到连续的数字和小数点、负号、科学计数法组成的字符串,则将字符串转化为数字后压入栈中

	4.如果遇到了运算符,则将栈顶的两个数字弹出,进行运算,运算后将结果压入栈中
	
	5.运算结束后数字栈中仅有一个数字,即为计算结果,返回此数字,完成计算	
	\subsection{主函数的设计}
	
	\subsubsection{main()函数}
	
	我设计了一个主函数,使用一个for循环,让用户输入字符串作为测试样例。首先用islegal()函数判断用户输入的字符串的合法性,如果不合法则直接输出"ILLEGAL";如果合法,则调用change()函数将中缀表达式转化为后缀表达式,输出后缀表达式,然后调用solution()函数计算后缀表达式的值。
	
	\subsection{基于负数和科学计数法的考虑}
	
	1.首先在legal()函数中,遇到某运算符和减号的连续运算不会返回0,而是将减号认为成负号
	
	2.在change()函数中,如果输入的是数字,则考虑科学计数法,将数字与e和e后面的幂次直接加到后缀表达式
	
	3.如果输入的是左括号或运算符且其后面有负号,则将运算符压栈并将负号加到后缀表达式并跳过处理减号
	
	4.在solution()函数中如果遇到负数或者科学计数法则直接压栈,进行后续计算即可
	
	\section{测试的结果}
	
	\begin{lstlisting}
		root@tian:~/.vscode/evl# g++ main.cpp -o main
		root@tian:~/.vscode/evl# ./main
		Please put a string as a test1
		1++1
		ILLEGAL
		Please put a string as a test2
		(+2-1)  
		ILLEGAL
		Please put a string as a test3
		3(2+1)
		ILLEGAL
		Please put a string as a test4
		((3+4)
		ILLEGAL
		Please put a string as a test5
		-3.1415926/0*2
		ILLEGAL
		Please put a string as a test6
		+12-1
		ILLEGAL
		Please put a string as a test7
		(3-2)*(4-3)/(2-5)
		after transfering:
		3 2 - 4 3 - * 2 5 - / 
		the result of7
		-0.333333
		Please put a string as a test8
		((3+-2)+(4-2))*((3-2)-4)       
		after transfering:
		3 -2 + 4 2 - + 3 2 - 4 - * 
		the result of8
		-9
		Please put a string as a test9
		3+1/2*4-2                           
		after transfering:
		3 1 2 / 4 * + 2 - 
		the result of9
		3
		Please put a string as a test10
		(-3.1415926+-2.71828)*2e2/3*3e-1+1-2
		after transfering:
		-3.1415926 -2.71828 + 2e2 * 3 / 3e-1 * 1 + 2 - 
		the result of10
		-118.197
	\end{lstlisting}
	
	前六次测试为表达式非法性测试,后四次测试则为合法性复杂表达式计算结果测试。
	
	测试结果一切正常。
	
	四则混合运算器设计成功。
	
	我用 valgrind 进行测试,发现没有发生内存泄露。
	
\end{document}

%%% Local Variables: 
%%% mode: latex
%%% TeX-master: t
%%% End: 
